\chapter{Описание API}

\section{MyStrategy}

В пакете для вашего языка программирования вы можете найти файл \texttt{MyStrategy/my\_strategy.<ext>}.
Этот файл содержит класс \texttt{MyStrategy} с методом \texttt{get\_action}, где должна быть реализована логика вашей стратегии.

Этот метод будет вызываться каждый тик, отдельно для каждого из ваших юнитов.

Метод принимает следующие аргументы:

\begin{itemize}
    \item Текущий юнит для которого надо вычислить действие (объект $Unit$)
    \item Текущее состояние игрового мира (объект $Game$)
    \item Вспомогательный отладочный объект (объект $Debug$).
        Этот объект позволяет вам отрисовывать отладочную информацию прямо из кода вашей стратегии.
        Заметьте, что использование этого объекта не имеет эффекта при тестировании на сервере,
        поэтому использовать его стоит только для локальной отладки.
\end{itemize}

Метод должен вернуть объект $UnitAction$, определяющий действие для заданного юнита.

\section{Описание объектов}

В этой секции, некоторые поля могут быть опциональными (обозначается как $Optional \langle type \rangle$).
Способ реализации зависит от языка.
При возможности используется специальный опциональный (nullable) тип,
иначе другие методы могут быть использованы (например nullable указатели).

Некоторые объекты могут принимать несколько различных форм. Способ реализации зависит от языка.
Если возможно, используется специальный (алгебраический) тип данных,
иначе другие методы могут быть использованы (например варианты представлены классами унаследованными от абстрактного базового класса).

$float32$ --- 32-битный вещественный тип данных, может называться $float$, и $float64$ может называться $double$ в некоторых языках.

\subsection{Bullet}
Определяет пулю. Поля:
\begin{itemize}
    \item $weapon\_type : WeaponType$ --- тип оружия из которого была выстрелена пуля
    \item $unit\_id : int$ --- id выстрелившего юнита
    \item $player\_id : int$ --- id игрока выстрелившего юнита
    \item $position : Vec2 \langle float64 \rangle$ --- позиция пули (центр)
    \item $velocity : Vec2 \langle float64 \rangle$ --- скорость пули (в единицах в секунду)
    \item $damage : int$ --- урон наносимый юниту при попадании
    \item $size : float64$ --- размер пули (сторона квадрата определяющего пулю)
    \item $explosion\_params : Optional \langle ExplosionParams \rangle$ --- параметры взрыва, если применимо
\end{itemize}

\subsection{BulletParams}
Параметры определяющие способ создания пули при выстреле. Поля:
\begin{itemize}
    \item $speed : float64$ --- скорость (длина вектора скорости)
    \item $size : float64$ --- размер
    \item $damage : int$ --- урон
\end{itemize}

\subsection{Color}
Цвет (используется для отладочной отрисовки). Поля:
\begin{itemize}
    \item $r : float32$ --- красная компонента
    \item $g : float32$ --- зеленая компонента
    \item $b : float32$ --- синяя компонента
    \item $a : float32$ --- альфа компонента (непрозрачность)
\end{itemize}

\subsection{ColoredVertex}
Вершина полигона (используется для отладочной отрисовки). Поля:
\begin{itemize}
    \item $position : Vec2 \langle float32 \rangle$ --- положение
    \item $color : Color$ --- цвет
\end{itemize}

\subsection{CustomData}
Тип данных используемый для отправки через объект $Debug$ (используется для отладочной отрисовки).
Может принимать следующие формы:
\begin{itemize}
    \item $Log$ --- логирование. Поля:
        \begin{itemize}
            \item $text : string$ --- текст для отображения
        \end{itemize}
    \item $Rect$ --- прямоугольник. Поля:
        \begin{itemize}
            \item $pos : Vec2 \langle float32 \rangle$ --- положение (левый нижний угол)
            \item $size : Vec2 \langle float32 \rangle$ --- размер
            \item $color : Color$ --- цвет для заполнения
        \end{itemize}
    \item $Line$ --- отрезок. Поля:
        \begin{itemize}
            \item $p1 : Vec2 \langle float32 \rangle$ --- первая точка
            \item $p2 : Vec2 \langle float32 \rangle$ --- вторая точка
            \item $width : float32$ --- ширина
            \item $color : Color$ --- цвет
        \end{itemize}
    \item $Polygon$ --- выпуклый полигон.
        Каждая вершина может быть покрашена отдельно --- цвет будет интерполирован между вершинами. Поля:
        \begin{itemize}
            \item $vertices : List \langle ColoredVertex \rangle$ --- список вершин
        \end{itemize}
    \item $PlacedText$ --- текст. Поля:
        \begin{itemize}
            \item $text : string$ --- текст для отображения
            \item $pos : Vec2 \langle float32 \rangle$ --- позиция (в игровых координатах)
            \item $alignment : TextAlignment$ --- способ выравнивания текста
            \item $size : float32$ --- размер текста в пикселях
            \item $color: Color$ --- цвет
        \end{itemize}
\end{itemize}

\subsection{ExplosionParams}
Параметры взрыва от мины или от взрывной пули. Поля:
\begin{itemize}
    \item $radius : float64$ --- радиус взрыва (половина стороны квадрата области взрыва)
    \item $damage : int$ --- урон
\end{itemize}

\subsection{Game}
Текущее состояние игры. Поля:
\begin{itemize}
    \item $current\_tick : int$ --- индекс текущего тика
    \item $properties : Properties$ --- параметры игры (константы)
    \item $level : Level$ --- уровень (карта)
    \item $players : List \langle Player \rangle$ --- список игроков (стратегий), участвующих в игре
    \item $units : List \langle Unit \rangle$ --- список живых юнитов
    \item $bullets : List \langle Bullet \rangle$ --- список летящих пуль
    \item $mines : List \langle Mine \rangle$ --- список \textbf{установленных} мин
    \item $loot\_boxes : List \langle LootBox \rangle$ --- список объектов лута
\end{itemize}

\subsection{Item}
Предмет, являющийся лутом.
Может принимать следующие формы
\begin{itemize}
    \item $HealthPack$ --- аптечка. Поля:
        \begin{itemize}
            \item $health : int$ --- пополняемое здоровье
        \end{itemize}
    \item $Weapon$ --- оружие. Поля:
        \begin{itemize}
            \item $weapon\_type : WeaponType$ --- тип оружия
        \end{itemize}
    \item $Mine$ --- мина. Без полей.
\end{itemize}

\subsection{JumpState}
Состояние прыжка юнита. Поля:
\begin{itemize}
    \item $can\_jump : boolean$ --- может ли юнит начать/продолжить прыжок
    \item $speed : float64$ --- скорость прыжка (в единицах в секунду)
    \item $max\_time : float64$ --- максимальная длительность прыжка (в секундах)
    \item $can\_cancel : boolean$ --- можно ли отменить текущий прыжок
\end{itemize}

\subsection{Level}
Уровень (карта). Поля:
\begin{itemize}
    \item $tiles : List \langle List \langle Tile \rangle \rangle$ --- двумерный список тайлов уровня
\end{itemize}

\subsection{LootBox}
Лут. Поля:
\begin{itemize}
    \item $position : Vec2 \langle float64 \rangle$ --- положение (середина нижней границы)
    \item $size : Vec2 \langle float64 \rangle$ --- размер
    \item $item : Item$ --- предмет
\end{itemize}

\subsection{Mine}
\textbf{Установленная} мина. Поля:
\begin{itemize}
    \item $player\_id : int$ --- id игрока, установившего мину
    \item $position : Vec2 \langle float64 \rangle$ --- положение (середина нижней границы)
    \item $size : Vec2 \langle float64 \rangle$ --- размер
    \item $state : MineState$ --- текущее состояние
    \item $timer : Optional \langle float64 \rangle$ --- время до смены состояния. Может отсутствовать
    \item $trigger\_radius : float64$ --- радиус запуска
    \item $explosion\_params : ExplosionParams$ --- параметры взрыва
\end{itemize}

\subsection{MineState}
Состояние мины. Варианты:
\begin{itemize}
    \item Preparing --- Подготовка
    \item Idle --- Ожидание
    \item Triggered --- Запуск
\end{itemize}

\subsection{Player}
Игрок (стратегия), участвующая в игре. Поля:
\begin{itemize}
    \item $id : int$
    \item $score : int$ --- текущий счет
\end{itemize}

\subsection{Properties}
Параметры игры (константы). Поля:
\begin{itemize}
    \item $max\_tick\_count : int$ --- максимальная длительность игры в тиках
    \item $ticks\_per\_second : float64$ --- количество тиков в ``секунду''
    \item $updates\_per\_tick : int$ --- количество обновлений в тик.
        Каждое обновление продвигает игровое время на $\frac{1}{ticks\_per\_second \times updates\_per\_tick}$
    \item $loot\_box\_size : Vec2 \langle float64 \rangle$ --- размер коробок с лутом
    \item $unit\_size : Vec2 \langle float64 \rangle$ --- размер юнитов
    \item $unit\_max\_horizontal\_speed : float64$ --- максимальная горизонтальная скорость юнитов
    \item $unit\_fall\_speed : float64$ --- скорость падения юнитов
    \item $unit\_jump\_time : float64$ --- обычное время прыжка
    \item $unit\_jump\_speed : float64$ --- обычная скорость прыжка
    \item $jump\_pad\_jump\_time : float64$ --- время прыжка с трамплина
    \item $jump\_pad\_jump\_speed : float64$ --- скорость прыжка с трамплина
    \item $unit\_max\_health : int$ --- максимальное (стартовое) здоровье юнита
    \item $weapon\_params : Map \langle WeaponType \rightarrow WeaponParams \rangle$ --- параметры для каждого типа оружия
    \item $mine\_size : Vec2 \langle float64 \rangle$ --- размер \textbf{установленных} мин
    \item $mine\_explosion\_params : ExplosionParams$ --- параметры взрыва мин
    \item $mine\_prepare\_time : float64$ --- время подготовки мины
    \item $mine\_trigger\_time : float64$ --- время запуска мины
    \item $mine\_trigger\_radius : float64$ --- радиус запуска мины
    \item $kill\_score : int$ --- очки за уничтожение вражеского юнита
\end{itemize}

\subsection{TextAlignment}
Выравнивание текста (используется для отладочной отрисовки). Варианты:
\begin{itemize}
    \item Left --- выравнивание по левому краю
    \item Center --- выравнивание по центру
    \item Right --- выравнивание по правому краю
\end{itemize}

\subsection{Tile}
Тайл уровня. Варианты:
\begin{itemize}
    \item Empty --- пусто
    \item Wall --- стена
    \item Platform --- платформа
    \item Ladder --- лестница
    \item JumpPad --- трамплин
\end{itemize}

\subsection{Unit}
Юнит. Поля:
\begin{itemize}
    \item $player\_id : int$ --- id игрока
    \item $id : int$ --- id юнита
    \item $health : int$ --- здоровье
    \item $position : Vec2 \langle float64 \rangle$ --- позиция (середина нижней границы)
    \item $size : Vec2 \langle float64 \rangle$ --- размер
    \item $jump\_state : JumpState$ --- текущее состояние прыжка
    \item $mines : int$ --- количество мин в инвентаре
    \item $weapon : Optional \langle Weapon \rangle$ --- текущее оружие, если есть
\end{itemize}

\subsection{UnitAction}
Действие юнита. Поля:
\begin{itemize}
    \item $velocity : float64$ --- целевая горизонтальная скорость
    \item $jump : boolean$ --- начать/продолжить прыжок/вверх по лестнице
    \item $jump\_down : boolean$ --- спрыгивать с/сквозь платформы/вниз по лестнице
    \item $aim : Vec2 \langle float64 \rangle$ --- направление прицеливания. Игнорируется если длина меньше $0.5$
    \item $shoot : boolean$ --- контролирует стрельбу/перезарядку
    \item $swap\_weapon : boolean$ --- поменять оружие
    \item $plant\_mine : boolean$ --- установить мину
\end{itemize}

\subsection{Vec2}
Двумерный вектор. Поля:
\begin{itemize}
    \item $x : float$
    \item $y : float$
\end{itemize}

\subsection{Weapon}
Оружие в руках юнита. Поля:
\begin{itemize}
    \item $type : WeaponType$ --- тип оружия
    \item $params : WeaponParams$ --- параметры (константы)
    \item $magazine : int$ --- текущая обойма
    \item $spread : float64$ --- текущее значение разброса
    \item $fire\_timer : Optional \langle float64 \rangle$ --- время до следующего выстрела (отсутствует если выстрел возможен)
    \item $last\_angle : Optional \langle float64 \rangle$ --- последний угол прицеливания (отсутствует при подборе)
\end{itemize}

\subsection{WeaponParams}
Параметры оружия (константы). Поля:
\begin{itemize}
    \item $magazine\_size : int$ --- максимальный размер обоймы
    \item $fire\_rate : float64$ --- время между выстрелами
    \item $reload\_time : float64$ --- время перезарядки
    \item $min\_spread : float64$ --- минимальный разброс
    \item $max\_spread : float64$ --- максимальный разброс
    \item $recoil : float64$ --- значение отдачи
    \item $aim\_speed : float64$ --- скорость прицеливания
    \item $bullet : BulletParams$ --- параметры пуль
    \item $explosion : Optional \langle ExplosionParams \rangle$ --- параметры взрыва пуль (если применимо)
\end{itemize}

\subsection{WeaponType}
Варианты:
\begin{itemize}
    \item Pistol
    \item AssaultRifle
    \item RocketLauncher
\end{itemize}